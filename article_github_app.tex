\documentclass[12pt,a4paper]{article}
\usepackage[russian]{babel}
\usepackage[utf8]{inputenc}
\begin{document}
Надеюсь, что мы (имеется ввиду автор и читатели) являемся последовательными поклонниками
интерфейса командной строки. Однако, отрицать право на существование графического интерфейса было бы
глубоко ошибочным.

В этой статье мы рассмотрим и опробуем графическую программу, позволяющую понятно работать
с git и github.com. Встречайте: \texttt{GitHub Desktop}.

\section{Установка.}
Откройте \texttt{https://desktop.github.com/download/} и нажмите кнопку \texttt{Download for macOS}
 или, что можно представить с трудом, \texttt{Download fo Windows}.

Все! Для установки больше ничего не нужно делать. Всё само сделается, и среди ваших приложения
появится рассматриваемая далее программа.
\section{Настройка.}

\end{document}
